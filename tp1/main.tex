\documentclass[titlepage,a4paper]{article}

\usepackage{a4wide}
\usepackage[colorlinks=true,linkcolor=black,urlcolor=blue,bookmarksopen=true]{hyperref}
\usepackage{bookmark}
\usepackage{fancyhdr}
\usepackage{amsmath}
\usepackage[spanish]{babel}
\usepackage[utf8]{inputenc}
\usepackage[T1]{fontenc}
\usepackage{graphicx}
\usepackage{float}
\usepackage{parskip}
\pagestyle{fancy} % Encabezado y pie de página
\fancyhf{}
\fancyhead[L]{TP1}
\fancyhead[R]{Teoría de Algoritmos}
\renewcommand{\headrulewidth}{0.4pt}
\fancyfoot[R]{\thepage}
\fancyfoot[L]{Melina Lazzaro}
\renewcommand{\footrulewidth}{0.4pt}

\newcommand{\HRule}{\rule{\linewidth}{0.5mm}}

\begin{document}
\begin{titlepage} % Carátula
	\includegraphics[width=0.75\textwidth]{logofiuba.jpg}\\[4cm] 
    \centering
    \textsc{\LARGE Teoría de Algoritmos I}\\[0.5cm]
    \textsc{\Large Segundo cuatrimestre de 2021 }\\[0.5cm]  
    \HRule \\[0.4cm]
    {\huge \bfseries Trabajo Práctico 1}\\[0.3cm]
    \HRule \\[2cm]
  	\Large
  	\begin{tabular}{ | l | l | l | }
  	    \hline
  	     Alumno & Número de padrón & Email \\ \hline
  	     LAZZARO, Melina & 105931 & mlazzaro@fi.uba.ar\\
  	     \hline
  	\end{tabular}
  	\vfill
            {\Large Profesor: Podberezski, Víctor Daniel}\\
  	\vfill
  	{\large Entrega: 29 de Septiembre de 2021}
\end{titlepage}

\tableofcontents % Índice general
\newpage

\section{Karatsuba}
\subsection{Multiplicación paso a paso mediante el algoritmo de Karatsuba}

Dados dos números  \textbf{$x=13594113$}  e  \textbf{$y=23985455$}  y  $n=8$  su número de dígitos 
El algoritmo de Karatsuba consiste en encontrar $x\cdot y$ separando los números por la mitad, pudiendo así expresarlos de la siguiente manera:
\begin{gather*} 
x = x_1 10^{n/2} + x_0  \qquad y = y_1 10^{n/2} + y_0 
\end{gather*}
\\
Es decir, el método realiza llamadas recursivas hasta que al menos uno de los números recibidos tenga un solo dígito.

Luego, 
\begin{gather*}
    x\cdot y=(x_110^{n/2}+x_0)(y_110^{n/2}+y_0)=  \\
=(x_110^{n/2})(y_110^{n/2})+x_1y_010^{n/2}+x_0y_110^{n/2}+x_0y_0 \\
=x_1y_110^{2(n/2)}+(x_1y_0+x_0y_1)10^{n/2}+x_0y_0 
\end{gather*}

Donde puedo calcular $x_1y_1$ y $x_0y_0$ mediante el algoritmos de Karatsuba, y $(x_1y_0+x_0y_1)$ como $((x_1+x_0)(y_1+y_0))-x_1y_1-x_0y_0$  \\ \\


\textbf{Inicialmente} \par
\qquad $x_1=1359$ \quad $x_0=4113$ \quad $y_1=2398$ \quad $y_0=5455$ \\
$x\cdot y= 1359 \cdot 2398 \cdot 10^8 + (1359\cdot 5455 + 4113 \cdot 2398)\cdot 10^4 + 4113\cdot 5455$ \qquad [1]\\

$1359 \cdot5455 + 4113\cdot2398 = ((1359+4113)\cdot (2398+5455)) - (1359\cdot 2398) - (4113\cdot 5455) = (5472\cdot7853) - (1359\cdot 2398) - (4113\cdot 5455) $ \\

Reemplazando en [1]: \qquad $ x\cdot y= 1359 \cdot 2398 \cdot 10^8 + ( 5472\cdot7853 - 1359\cdot 2398 - 4113\cdot 5455)\cdot 10^4 + 4113\cdot 5455 $ \\ \\





\textbf{Karatsuba con  1359x2398} \par
\qquad $x_1=13$ \quad $x_0=59$ \quad $y_1=23$ \quad $y_0=98$ \\
$x\cdot y= 13 \cdot 23 \cdot 10^4 + (13\cdot 98 + 59 \cdot 23)\cdot 10^2 + 59\cdot 98$ \qquad [2] \\ 

$13\cdot 98 + 59 \cdot 23 = ((13+59)\cdot (23+98)) - (13\cdot 23) - (59\cdot 98) = (72 \cdot 121) - (13\cdot 23) - (59\cdot 98) $ \\

Reemplazando en [2]: \qquad $ x\cdot y= 13 \cdot 23 \cdot 10^4 + ((72 \cdot 121) - (13\cdot 23) - (59\cdot 98))\cdot 10^2 + 59\cdot 98 $ \\ \\ 


\textbf{Karatsuba con 13x23} \par
\qquad $x_1=1$ \quad $x_0=3$ \quad $y_1=2$ \quad $y_0=3$ \\
$x\cdot y= 1 \cdot 2 \cdot 10^2 + (1\cdot 3 + 3 \cdot 2)\cdot 10^1 + 3\cdot 3$ \\

La llamada a Karatsuba con 1,2 y 3,3 devolverá el producto de los números ya que tienen un solo dígito, entonces: \\ \\
$x\cdot y= 2 \cdot 10^2 + (1\cdot 3 + 3 \cdot 2)\cdot 10^1 + 9$ \qquad [3]\\ 

$1 \cdot3 + 3\cdot2 = ((1+3)\cdot (2+3)) - (1\cdot 2) - (3\cdot 3)$  de donde ya se calculó $1\cdot 2$ y $3\cdot 3$, $(1+3)\cdot (2+3) = 4\cdot5$ y como tienen un solo dígito la llamada a Karatsuba devuelve el producto de los mismos $4\cdot5=20$, por lo que $ (1\cdot 3 + 3 \cdot 2) = 20 - 2 - 9 =9 $ \\ \\ 
Reemplazando en [3]: $x\cdot y= 2 \cdot 10^2 + 9\cdot 10^1 + 9 = 299$ \\ \\


\textbf{Karatsuba con 59x98} \par
\qquad $x_1=5$ \quad $x_0=9$ \quad $y_1=9$ \quad $y_0=8$ \\
$x\cdot y= 5 \cdot 9 \cdot 10^2 + (5\cdot 8 + 9 \cdot 9)\cdot 10^1 + 9\cdot 8$ \\ 

La llamada a Karatsuba con 5,9 y 9,8 devolverá el producto de los números ya que tienen un solo dígito, entonces: \\ \\
$x\cdot y= 45 \cdot 10^2 + (5\cdot 8 + 9 \cdot 9)\cdot 10^1 + 72$ \qquad [4]\\ 

$5 \cdot8 + 9\cdot9 = (5+9)\cdot (9+8) - 5\cdot 9 - 9\cdot 8$ de donde ya se calculó $5\cdot 9$ y $9\cdot 8$, $(5+9)\cdot (9+8) = 14\cdot17$, obteniendo: \\ \\
$ (5\cdot 8 + 9 \cdot 9) = 14\cdot17 - 45 - 72 $ \\ 

Reemplazando en [4]: \qquad $x\cdot y= 45 \cdot 10^2 + (14\cdot17 - 45 - 72)\cdot 10^1 + 72$ \\ \\


\textbf{Karatsuba con 14x17} \par
\qquad $x_1=1$ \quad $x_0=4$ \quad $y_1=1$ \quad $y_0=7$ \\
$x\cdot y= 1 \cdot 1 \cdot 10^2 + (1\cdot 7 + 4 \cdot 1)\cdot 10^1 + 4\cdot 7$ \\ 

La llamada a Karatsuba con 1,1 y 4,7 devolverá el producto de los números ya que tienen un solo dígito, entonces: \\ \\
$x\cdot y= 1 \cdot 10^2 + (1\cdot 7 + 4 \cdot 1)\cdot 10^1 + 28$ \qquad [5]\\ 

$1 \cdot7 + 4\cdot1 = ((1+4)\cdot (1+7)) - (1\cdot 1) - (4\cdot 7)$  de donde ya se calculó $1\cdot 1$ y $4\cdot 7$, $(1+4)\cdot (1+7) = 5\cdot8$ y como tienen un solo dígito la llamada a Karatsuba devuelve el producto de los mismos $5\cdot8=40$, por lo que $ (1\cdot 7 + 4 \cdot 1) = 40 - 1 - 28 = 11 $ \\ \\
Volviendo a [5]: $x\cdot y= 1 \cdot 10^2 + (1\cdot 7 + 4 \cdot 1)\cdot 10^1 + 28 = 1 \cdot 10^2 + 11\cdot 10^1 + 28 = 238$ \\ \\


\textbf{Volviendo a [4]} \par
Por [5]: \qquad $x\cdot y= 45 \cdot 10^2 + (14\cdot17 - 45 - 72)\cdot 10^1 + 72 = 45 \cdot 10^2 + (238 - 45 - 72)\cdot 10^1 + 72 = 5782 $ \\ \\


\textbf{Karatsuba con 072x121} \par
\qquad $x_1=07$ \quad $x_0=2$ \quad $y_1=12$ \quad $y_0=1$ \\
$x\cdot y= 7 \cdot 12 \cdot 10^2 + (7\cdot 1 + 2 \cdot 12)\cdot 10^1 + 2\cdot 1$ \\

La llamada a Karatsuba con 7,12 y 2,1 devolverá el producto de los números ya que tienen un solo dígito, entonces: \\ \\
$x\cdot y= 84\cdot 10^2 + (7\cdot 1 + 2 \cdot 12)\cdot 10^1 + 2 $ \qquad [6] \\

$7 \cdot1 + 2\cdot12 = ((7+2)\cdot (12+1)) - (7\cdot 12) - (2\cdot 1)$  de donde ya se calculó $7\cdot 12$ y $2\cdot 1$, $(7+2)\cdot (12+1) = 9\cdot13$ y como tienen un solo dígito la llamada a Karatsuba devuelve el producto de los mismos $9\cdot13=117$, por lo que $ (7 \cdot1 + 2\cdot12) = 117 - 84 - 2 = 31 $ \\ \\
Volviendo a [6]: $x\cdot y= 84\cdot 10^2 + (7\cdot 1 + 2 \cdot 12)\cdot 10^1 + 2  = 84 \cdot 10^2 + 31\cdot 10^1 + 2 = 8712$ \\ \\


\textbf{Volviendo a [2]} \par
Por [3], [4] y [6]: \qquad $x\cdot y= 299 \cdot 10^4 + (8712 - 299 - 5782)\cdot 10^2 + 5782 = 3258882$ \\ \\





\textbf{Karatsuba con  4113x5455} \par
\qquad $x_1=41$ \quad $x_0=13$ \quad $y_1=54$ \quad $y_0=55$ \\
$x\cdot y= 41 \cdot 54 \cdot 10^4 + (41\cdot 55 + 13 \cdot 54)\cdot 10^2 + 13\cdot 55$ \qquad [7] \\

$41\cdot 55 + 13 \cdot 54 = ((41+13)\cdot (54+55)) - (41\cdot 54) - (13\cdot 55) = (54 \cdot 109) - (41\cdot 54) - (13\cdot 55) $ \\

Reemplazando en [7]: \qquad $x\cdot y= 41 \cdot 54 \cdot 10^4 + ((54 \cdot 109) - (41\cdot 54) - (13\cdot 55))\cdot 10^2 + 13\cdot 55$ \\ \\ 


\textbf{Karatsuba con 41x54} \par
\qquad $x_1=4$ \quad $x_0=1$ \quad $y_1=5$ \quad $y_0=4$ \\
$x\cdot y= 4 \cdot 5 \cdot 10^2 + (4\cdot 4 + 1 \cdot 5)\cdot 10^1 + 1\cdot 4$ \\

La llamada a Karatsuba con 4,5 y 1,4 devolverá el producto de los números ya que tienen un solo dígito, entonces: \\ \\
$x\cdot y= 20 \cdot 10^2 + (4\cdot 4 + 1 \cdot 5)\cdot 10^1 + 4$ \qquad [8]\\ 

$4 \cdot4 + 1\cdot5 = ((4+1)\cdot (5+4)) - (4\cdot 5) - (1\cdot 4)$ de donde ya se calculó $4\cdot 5$ y $1\cdot 4$, $(4+1)\cdot (5+4) = 5\cdot9$ y como tienen un solo dígito la llamada a Karatsuba devuelve el producto de los mismos $5\cdot9=45$, por lo que $(4\cdot 4 + 1 \cdot 5) = 45 - 20 - 4 = 21 $ \\ \\ 
Reemplazando en [8]: \qquad $ x\cdot y= 20 \cdot 10^2 + 21\cdot 10^1 + 4 = 2214$ \\ \\ \\ 


\textbf{Karatsuba con 13x55} \par
\qquad $x_1=1$ \quad $x_0=3$ \quad $y_1=5$ \quad $y_0=5$ \\
$x\cdot y= 1 \cdot 5 \cdot 10^2 + (1\cdot 5 + 3 \cdot 5)\cdot 10^1 + 3\cdot 5$ \\

La llamada a Karatsuba con 1,5 y 3,5 devolverá el producto de los números ya que tienen un solo dígito, entonces: \\ \\
$x\cdot y= 5 \cdot 10^2 + (1\cdot 5 + 3 \cdot 5)\cdot 10^1 + 15$ \qquad [9]\\

$1 \cdot5 + 3\cdot5 = ((1+3)\cdot (5+5)) - (1\cdot 5) - (3\cdot 5)$ de donde ya se calculó $1\cdot 5$ y $3\cdot 4$, $(1+3)\cdot (5+5) = 4\cdot10$ y como tienen un solo dígito la llamada a Karatsuba devuelve el producto de los mismos $4\cdot10=40$, por lo que $(1\cdot 5 + 3 \cdot 5) = 40 - 5 - 15 = 20 $ \\ \\ 
Reemplazando en [9]: \qquad $x\cdot y= 5 \cdot 10^2 + 20\cdot 10^1 + 15 = 715$ \\ \\


\textbf{Karatsuba con 054x109} \par
\qquad $x_1=05$ \quad $x_0=4$ \quad $y_1=10$ \quad $y_0=9$ \\
$x\cdot y= 05 \cdot 10 \cdot 10^2 + (5\cdot 9 + 4 \cdot 10)\cdot 10^1 + 4\cdot 9$ \\

La llamada a Karatsuba con 5,10 y 4,9 devolverá el producto de los números ya que tienen un solo dígito, entonces: \\ \\
$x\cdot y= 50 \cdot 10^2 + (5\cdot 9 + 4 \cdot 10)\cdot 10^1 + 36$ \qquad [10] \\

$5 \cdot9 + 4\cdot10 = ((5+4)\cdot (10+9)) - (5\cdot 10) - (4\cdot 9)$  \\ de donde ya se calculó $5\cdot 10$ y $4\cdot 9$, $(5+4)\cdot (10+9) = 9\cdot19$ y como tienen un solo dígito la llamada a Karatsuba devuelve el producto de los mismos $9\cdot19=171$, por lo que $(5\cdot 9 + 4 \cdot 10) = 171 - 50 - 36 = 85 $ \\ \\ 
Reemplazando en [10]: \qquad $x\cdot y= 50 \cdot 10^2 + 85 \cdot 10^1 + 36 = 5886$ \\ \\


\textbf{Volviendo a [7]} \par
Por [8], [9] y [10]: \qquad $x \cdot y = 2214 \cdot 10^4 + (5886 - 2214 -715) \cdot 10^2 + 715 = 22436415 $ \\ \\





\textbf{Karatsuba con  5472x7853} \par
\qquad $x_1=54$ \quad $x_0=72$ \quad $y_1=78$ \quad $y_0=53$ \\
$x\cdot y= 54 \cdot 78 \cdot 10^4 + (54\cdot 53 + 72 \cdot 78)\cdot 10^2 + 72\cdot 53$ \qquad [11] \\

$54\cdot 53 + 72 \cdot 78 = ((54+72)\cdot (78+53)) - (54\cdot 78) - (72\cdot 53) = 126 \cdot 131 - 54\cdot 78 - 72\cdot 53 $ \\

Reemplazando en [11]: \qquad $ x\cdot y= 54 \cdot 78 \cdot 10^4 + (126 \cdot 131 - 54\cdot 78 - 72\cdot 53)\cdot 10^2 + 72\cdot 53 $ \\ \\ \\ \\


\textbf{Karatsuba con 54x78} \par
\qquad $x_1=5$ \quad $x_0=4$ \quad $y_1=7$ \quad $y_0=8$ \\
$x\cdot y= 5 \cdot 7 \cdot 10^2 + (5\cdot 8 + 4 \cdot 7)\cdot 10^1 + 4\cdot 8 $ \\

La llamada a Karatsuba con 5,7 y 4,8 devolverá el producto de los números ya que tienen un solo dígito, entonces: \\ \\
$x\cdot y = 35 \cdot 10^2 + (5\cdot 8 + 4 \cdot 7)\cdot 10^1 + 32 $ \qquad [12] \\

$5 \cdot8 + 4\cdot7 = ((5+4)\cdot (7+8)) - (5\cdot 7) - (4\cdot 8) $ de donde ya se calculó $5\cdot 7$ y $4\cdot 8$, $(5+4)\cdot (7+8) = 9\cdot15$ y como tienen un solo dígito la llamada a Karatsuba devuelve el producto de los mismos $9\cdot15=135$, por lo que $ (5\cdot 8 + 4 \cdot 7) = 135 - 35 - 32 = 68 $ \\ \\ 
Reemplazando en [12]: \qquad $ x\cdot y = 35 \cdot 10^2 + 68\cdot 10^1 + 32 = 4212 $ \\ \\


\textbf{Karatsuba con 72x53} \par
\qquad $x_1=7$ \quad $x_0=2$ \quad $y_1=5$ \quad $y_0=3$ \\
$x\cdot y= 7 \cdot 5 \cdot 10^2 + (7\cdot 3 + 2 \cdot 5)\cdot 10^1 + 2\cdot 3 $ \\

La llamada a Karatsuba con 7,5 y 2,3 devolverá el producto de los números ya que tienen un solo dígito, entonces: \\ \\
$x\cdot y = 35 \cdot 10^2 + (7\cdot 3 + 2 \cdot 5)\cdot 10^1 + 6 $ \qquad [13] \\

$7 \cdot3 + 2\cdot5 = ((7+2)\cdot (5+3)) - (7\cdot 5) - (2\cdot 3)$ de donde ya se calculó $7\cdot 5$ y $2\cdot 3$, $(7+2)\cdot (5+3) = 9\cdot8$ y como tienen un solo dígito la llamada a Karatsuba devuelve el producto de los mismos $9\cdot8=72$, por lo que $ (7\cdot 3 + 2 \cdot 5) = 72 - 35 - 6 = 31 $ \\ \\ 
Reemplazando en [13]: \qquad $ x\cdot y = 35 \cdot 10^2 + 31\cdot 10^1 + 6 = 3816 $ \\ \\


\textbf{Karatsuba con 126x131} \par
\qquad $x_1=12$ \quad $x_0=6$ \quad $y_1=13$ \quad $y_0=1$ \\
$x\cdot y= 12 \cdot 13 \cdot 10^2 + (12\cdot 1 + 6 \cdot 13)\cdot 10^1 + 6\cdot 1$ \\

La llamada a Karatsuba con 6,1 devolverá el producto de los números ya que tienen un solo dígito, entonces: \\ \\
$x\cdot y= 12 \cdot 13 \cdot 10^2 + (12\cdot 1 + 6 \cdot 13)\cdot 10^1 + 6 $ \qquad [14] \\

$12\cdot 1 + 6 \cdot 13 = ((12+6)\cdot (13+1)) - (12\cdot 13) - (6\cdot 1) $  de donde ya se calculó $6 \cdot 1$, $ (12+6)\cdot (13+1) = 18 \cdot 14 $, por lo que $(12\cdot 1 + 6 \cdot 13) =  18 \cdot 14 - (12\cdot 13) - 6 $ \\ \\
Reemplazando en [14]: \qquad $ x\cdot y= 12 \cdot 13 \cdot 10^2 + (18 \cdot 14 - (12\cdot 13) - 6 )\cdot 10^1 + 6  $ \\ \\ \\


\textbf{Karatsuba con 12x13} \par
\qquad $x_1=1$ \quad $x_0=2$ \quad $y_1=1$ \quad $y_0=3$ \\
$x\cdot y= 1 \cdot 1 \cdot 10^2 + (1\cdot 3 + 2 \cdot 1)\cdot 10^1 + 2\cdot 3 $ \\

La llamada a Karatsuba con 1,1 y 2,3 devolverá el producto de los números ya que tienen un solo dígito, entonces: \\ \\
$x\cdot y = 1 \cdot 10^2 + (1\cdot 3 + 2 \cdot 1)\cdot 10^1 + 6 $ \qquad [15] \\

$1 \cdot3 + 2\cdot1 = ((1+2)\cdot (1+3)) - (1\cdot 1) - (2\cdot 3) $ de donde ya se calculó $1 \cdot 1$ y $ 2 \cdot 3 $, $ (1+2)\cdot (1+3) = 3 \cdot 4 $y como tienen un solo dígito la llamada a Karatsuba devuelve el producto de los mismos $3\cdot4=12$, por lo que $ (1\cdot 3 + 2 \cdot 1) = 12 - 1 - 6 = 5 $ \\ \\ 
Reemplazando en [15]: \qquad $ x\cdot y= 1 \cdot 10^2 + 5\cdot 10^1 + 6 = 156 $ \\ \\


\textbf{Karatsuba con 18x14} \par
\qquad $x_1=1$ \quad $x_0=8$ \quad $y_1=1$ \quad $y_0=4$ \\
$x\cdot y= 1 \cdot 1 \cdot 10^2 + (1\cdot 4 + 8 \cdot 1)\cdot 10^1 + 8\cdot 4 $ \\

La llamada a Karatsuba con 1,1 y 8,4 devolverá el producto de los números ya que tienen un solo dígito, entonces: \\ \\
$x\cdot y = 1 \cdot 10^2 + (1\cdot 4 + 8 \cdot 1)\cdot 10^1 + 32 $ \qquad [16] \\

$1\cdot 4 + 8 \cdot 1 = ((1+8)\cdot (1+4)) - (1\cdot 1) - (8\cdot 4) $ de donde ya se calculó $1 \cdot 1$ y $ 8 \cdot 4 $, $ (1+8)\cdot (1+4) = 9 \cdot 5 $y como tienen un solo dígito la llamada a Karatsuba devuelve el producto de los mismos $9\cdot5=45$, por lo que $(1\cdot 4 + 8 \cdot 1) = 45 - 1 - 32 = 12 $ \\ \\ 
Reemplazando en [16]: \qquad $ x\cdot y= 1 \cdot 10^2 + 12\cdot 10^1 + 32 = 252 $ \\ \\


\textbf{Volviendo a [14]} \par
Por [15] y [16]: $x\cdot y= 12 \cdot 13 \cdot 10^2 + (12\cdot 1 + 6 \cdot 13)\cdot 10^1 + 6\cdot 1 = 156 \cdot 10^2 +  90\cdot 10^1 + 6 = 16506 $ \\ \\


\textbf{Volviendo a [11]} \par
Por [12], [13] y [14]: $ x\cdot y= 54 \cdot 78 \cdot 10^4 + (126 \cdot 131 - 54\cdot 78 - 72\cdot 53)\cdot 10^2 + 72\cdot 53 = 4212 \cdot 10^4 + (16506 - 4212 - 3816)\cdot 10^2 + 3816 = 42971616 $ \\ \\


\textbf{Finalmente, volviendo a [1]} \par
Por [2], [7] y [11]: $ x\cdot y= 1359 \cdot 2398 \cdot 10^8 + ( 5472\cdot7853 - 1359\cdot 2398 - 4113\cdot 5455)\cdot 10^4 + 4113\cdot 5455 = 3258882\cdot 10^8 + (42971616 -3258882-22436415) \cdot 10^4 +22436415 = 326060985626415$ \\ \\


\newpage
\subsection{Cuenta de sumas y multiplicaciones}

\textbf{Cantidad de multiplicaciones} \\
Para contar la cantidad de multiplicaciones tuve en cuenta la cantidad de llamadas a Karatsuba con el caso base (al menos un número de un dígito) y dos potencias de 10 por cada llamada que NO es un caso base.
\begin{itemize}
    \item Caso base: 33
    \item Potencias: 16 llamadas . 2 = 32
    \item Total: 65
\end{itemize}

\textbf{Cantidad de sumas}
Para contar la cantidad de sumas tuve en cuenta dos sumas por cada llamada a Karatsuba que NO es un caso base provenientes de $ x\cdot y=x_1y_110^{2(n/2)}+(x_1y_0+x_0y_1)10^{n/2}+x_0y_0 $,  y cuatro sumas (dos sumas y dos restas) también por cada llamada que no es un caso base provinientes de la fórmula: $(x_1y_0+x_0y_1)=((x_1+x_0)(y_1+y_0))-x_1y_1-x_0y_0$
\begin{itemize}
    \item Primera fórmula: 16 llamadas . 2 = 32
    \item Segunda fórmula: 16 llamadas . 4 = 64
    \item Total: 96
\end{itemize}

\textbf{Relación con la complejidad temporal} \\
La complejidad temporal está definida para $n\longrightarrow\infty$ y es $O(n^{log_2(3)})=O(n^{1,59})$ lo cual no se ve reflejado en este caso con n=8, ya que $8^{1,59}=27,28$ pero la cantidad de operaciones realizadas es mucho mayor. 


\subsection{Comparación con el método tradicional}
El método de división tradicional consiste en multiplicar cada dígito de uno de los números por cada uno de los dígitos del otro agregando un 0 cada vez para luego sumar todos los resultados. 
Este método es $O(n^2)$ y para este caso donde n=8, la cantidad de multiplicaciones que se deben realizar son $8^2=64$, menos que las realizadas mediante Karatsuba.\par 
Por lo tanto, se puede concluir que el método de Karatsuba no es conveniente para n pequeños.



\subsection{División y conquista}
El algoritmo de Karatsuba de puede considerar de división y conquista ya que \emph{divide} el problema inicial, una multiplicación de n dígitos, en multiplicaciones de n/2 dígitos recursivamente, \emph{conquista} cuando alcanza un caso base de una multiplicación de 1 dígito que se puede resolver fácilmente y finalmente \emph{combina} los resultados en una solución general. \par 
 
\newpage
\section{Cuestión de complejidad}

\subsection{Teorema maestro}
Dado:  $a T(n/b) + O(c)$ ; $a=2$; $b=5$; $c=n^2$ \\
Obtengo:
\begin{align*} 
2T(n/5)+O(n^2)
\end{align*}

Condiciones de la relación de recurrencia para aplicar el Teorema Maestro:
\begin{itemize}
    \item $a\geq 1$ y $b \geq 1$ constantes
    \item f(n) una función
    \item $T(n)=a T(n/b) + f(n)$ una realción de recurrencia con T(0)=cte
\end{itemize}

Lo que le falta para poder aplicar el teorema, es que T(0)=cte


\subsection{Complejidad temporal}

Para calcular la complejidad temporal, primero calculo $log_b(a)=log_5(2)=0,43$ y verifico a cual de los tres casos posibles le corresponde:
\begin{itemize}
    \item \textbf{Caso 1:} $f(n)=O(n^{log_b(a)-e})$ con e>0 \qquad $log_5(2)-e=0,43-e \neq 2$ \quad no cumple  
    \item \textbf{Caso 2:} $f(n)=\theta(n^{log_b(a)})$ \qquad $log_5(2)=0,43 \neq 2$ \quad no cumple   
    \item \textbf{Caso 3:}  $f(n)=\Omega(n^{log_b(a)+e})$ con e>0 \qquad $log_5(2)+e=0,43+e$ \quad cumple
\end{itemize}

Como estoy en el tercer caso, debo buscar una cota inferior tal que:
\begin{align*} 
    f(n)=\Omega(n^{log_b(a)+e}), \quad e>0 \quad \rightarrow  T(n)=\Theta(f(n)) \\
    E \quad c<1, n>> \quad / \quad af(n/b) \leq cf(n)
\end{align*}

Con $e=0,1$ se cumple que $n^2\geq\Omega(n^{0,53})$, luego:
\begin{align*}
     2f(n/5) \leq cf(n) \\
     2n^2/25 \leq cn^2
\end{align*}

Si $c=1/2$ la condición se cumple, por lo tanto $ T(n)=\Theta(n^2) $


\end{document}
